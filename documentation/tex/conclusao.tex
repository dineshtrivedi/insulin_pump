\chapter{Conclusão}

Este trabalho teve como base o desenvolvimento e simulação de um \emph{software} de controle para um protótipo de bomba de infusão de insulina baseado em um microcontrolador da família PIC. Além disso, desenvolveu-se juntamente com o software de controle um módulo de comunicação com periféricos para interação com o usuário. O \emph{software} desenvolvido é responsável por executar os perfis de infusão configurados pelo o usuário, ativar o motor de passos, dar \emph{feedback} do status da bomba e de sua execução ao paciente. Os objetivos desse trabalho foram alcançados ao desenvolver o \emph{software} utilizando o microcontrolador de baixo custo da família PIC: PIC18F452. 

A compreensão das tecnologias utilizadas, simulador e, é claro, da natureza do problema proposto foram imprescindíveis. O presente trabalho pode ser considerado uma prova de conceito do módulo desenvolvido, e pode ser continuado e aprimorado sem grandes problemas devido a sua alta modularidade e simplicidade do código. E com isso realizar testes mais precisos e factíveis com relação a realidade do problema proposto. 

Considerando como projeto futuro seria os testes utilizando um \emph{hardware}, placa da Microgenios citada anteriormente, composto pelos componentes abordados e comprovar as vantagens do uso do simulador, demonstrar que testes de \emph{stress}, ou seja, por longos períodos pode sem simulado no PC, graças ao OOC. E, por fim, talvez menos importante, conseguir isolar completamente a dependência do compilador e \emph{hardware} devido à forma de uso da interrupção para contagem de tempo para infusão.

Por fim, o código é aberto, licença MIT e pode ser encontrado em:  \url{https://gitlab.com/dinesh/insulin_pump}.